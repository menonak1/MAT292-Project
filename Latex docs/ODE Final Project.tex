%%%%%%%%%%%%%%%%%%%%%%%%%%%%%%%%%%%%%%%%%%%%%%%%%
% MAT292 Final Project Report Draft
% Second Draft - Key Equations Added
%%%%%%%%%%%%%%%%%%%%%%%%%%%%%%%%%%%%%%%%%%%%%%%%%
\documentclass[12pt, letterpaper]{article}

% --- PACKAGES ---
\usepackage[margin=1in]{geometry} % 1-inch margins
\usepackage{amsmath} % For math equations
\usepackage{amssymb} % For \mathbb symbols
\usepackage{graphicx} % For inserting figures
\usepackage{hyperref} % For clickable links
\usepackage{booktabs} % For nice tables (in results)
\usepackage[
backend=biber,
style=ieee, % A common engineering style for citations
sorting=none
]{biblatex}
\addbibresource{references.bib} % Use a separate .bib file

% --- TITLE INFO ---
\title{Attitude Control for Quadcopters: A PID-Based Approach}
\author{Akash Ajin, Akhilesh Menon, Paul Choi, Shiv Khannade\\ MAT292: Ordinary Differential Equations}
\date{Fall 2025}

\begin{document}

\maketitle

% --- 1. ABSTRACT ---
\begin{abstract}
This project investigated the attitude dynamics of a quadcopter, an unstable system
that requires active control for stable flight. We developed a mathematical model based on
Newton-Euler equations, designed a Proportional-Integral-Derivative (PID) control system to stabilize
it, and evaluated its performance through numerical simulation in Python.
This work demonstrates the practical application of solving systems of ordinary
differential equations (ODEs) to overcome real-world engineering challenges,
highlighting their importance.
\end{abstract}


% --- 2. INTRODUCTION ---
\section{Introduction}

\subsection{Motivation}
A quadcopter is an unmanned aerial vehicle (UAV) whose flight is controlled by
four motors. This UAV's navigational agility has made it popular
in fields from photography to logistics. However, quadcopters are
inherently unstable. Without a constant stream of adjustments
from a control system, minor disturbances would cause them to tumble.
The system's dynamics are also highly non-linear and coupled, making it a
very fitting problem for an ODE course.

\subsection{Project Goal}
The primary objective of this project was to model the unstable flight dynamics
of a quadcopter as a system of coupled, non-linear ordinary differential equations.
We then designed and simulated a PID feedback controller to impose stability,
allowing the UAV to achieve a stable hover and reject external disturbances.
The project provides a clear, practical demonstration of
how ODEs and control theory are used to solve a fundamental problem in modern
robotics.


% --- 3. MATHEMATICAL MODEL ---
\section{Mathematical Model and Theoretical Foundation}
The quadcopter's motion is modeled as a rigid body in 3D space, using a
fixed inertial frame (Earth, $E$) and a rotating body frame (Body, $B$) attached to the vehicle.

\subsection{State Vector}
The system is described by a 12-element state vector $\mathbf{x} \in \mathbb{R}^{12}$:
$$
\mathbf{x} = [
\underbrace{x, y, z}_{\substack{\text{position} \\ \text{(Frame E)}}},
\underbrace{\phi, \theta, \psi}_{\substack{\text{attitude (Euler)} \\ \text{(Frame E)}}},
\underbrace{\dot{x}, \dot{y}, \dot{z}}_{\substack{\text{lin. velocity} \\ \text{(Frame E)}}},
\underbrace{p, q, r}_{\substack{\text{ang. velocity} \\ \text{(Frame B)}}}
]^T
$$
This state is a hybrid, as linear motion is tracked in the inertial frame $E$
while rotational motion is tracked in the body frame $B$. This requires
transformation matrices to couple the dynamics.

\subsection{Translational Dynamics}
In the inertial frame, Newton's 2nd Law governs translational motion:
$$
\mathbf{F}_{\text{net}, E} = m \ddot{\mathbf{p}}_E = m \begin{bmatrix} \ddot{x} \\ \ddot{y} \\ \ddot{z} \end{bmatrix}
$$
The net force is the sum of gravity $\mathbf{F}_{grav, E} = [0, 0, -mg]^T$ and
the total thrust $\mathbf{F}_{thrust, B} = [0, 0, T]^T$. Thrust is generated
in the body frame (acting along the quadcopter's $z_B$-axis) and must be
rotated into the inertial frame:
$$
\begin{bmatrix} \ddot{x} \\ \ddot{y} \\ \ddot{z} \end{bmatrix} =
\frac{1}{m} \left(
\begin{bmatrix} 0 \\ 0 \\ -mg \end{bmatrix} +
R(\phi, \theta, \psi) \begin{bmatrix} 0 \\ 0 \\ T \end{bmatrix}
\right)
$$
where $T = T_1+T_2+T_3+T_4$ and $R$ is the $Z-Y-X$ body-to-inertial rotation matrix.
$R$ is constructed as $R = R_z(\psi) R_y(\theta) R_x(\phi)$:
$$
R =
\begin{bmatrix}
c\psi c\theta & c\psi s\theta s\phi - s\psi c\phi & c\psi s\theta c\phi + s\psi s\phi \\
s\psi c\theta & s\psi s\theta s\phi + c\psi c\phi & s\psi s\theta c\phi - c\psi s\phi \\
-s\theta & c\theta s\phi & c\theta c\phi
\end{bmatrix}
$$
where $c\alpha = \cos(\alpha)$ and $s\alpha = \sin(\alpha)$.
The first six ODEs are thus: $\dot{x} = \dot{x}$, $\dot{y} = \dot{y}$, $\dot{z} = \dot{z}$,
and the three translational acceleration equations above ($\ddot{x}, \ddot{y}, \ddot{z}$).

\subsection{Rotational Dynamics}
Rotational motion is described by the Newton-Euler equations in the body frame:
$$
\mathbf{\tau}_{B} = I \dot{\mathbf{\omega}}_B + \mathbf{\omega}_B \times (I \mathbf{\omega}_B)
$$
where $\mathbf{\tau}_B = [\tau_\phi, \tau_\theta, \tau_\psi]^T$ are the net torques,
$I$ is the $3 \times 3$ inertia tensor, and $\mathbf{\omega}_B = [p, q, r]^T$.
Assuming a diagonal inertia tensor $I = \text{diag}(I_{xx}, I_{yy}, I_{zz})$,
the ODEs for the angular rates are:
\begin{align*}
\dot{p} &= (1/I_{xx}) \left( \tau_\phi - (I_{zz} - I_{yy})qr \right) \\
\dot{q} &= (1/I_{yy}) \left( \tau_\theta - (I_{xx} - I_{zz})pr \right) \\
\dot{r} &= (1/I_{zz}) \left( \tau_\psi - (I_{yy} - I_{xx})pq \right)
\end{align*}
The Euler angle derivatives (in frame E) are related to the body rates (in frame B)
by the transformation $W$:
$$
\begin{bmatrix} \dot{\phi} \\ \dot{\theta} \\ \dot{\psi} \end{bmatrix} =
W(\phi, \theta) \begin{bmatrix} p \\ q \\ r \end{bmatrix}
=
\begin{bmatrix}
1 & \sin\phi \tan\theta & \cos\phi \tan\theta \\
0 & \cos\phi & -\sin\phi \\
0 & \sin\phi / \cos\theta & \cos\phi / \cos\theta
\end{bmatrix}
\begin{bmatrix} p \\ q \\ r \end{bmatrix}
$$
These nine equations form the complete system of 12 first-order ODEs
($\dot{x}, \dot{y}, \dot{z}, \dot{\phi}, \dot{\theta}, \dot{\psi}, \ddot{x}, \ddot{y}, \ddot{z}, \dot{p}, \dot{q}, \dot{r}$),
which can be written in the form $\dot{\mathbf{x}} = f(t, \mathbf{x}, \mathbf{u})$.


% --- 4. METHODOLOGY ---
\section{Methodology}

\subsection{Control Strategy: PID Controller}
To stabilize the unstable dynamics, we implemented four independent
PID controllers. The PID control law calculates a corrective action $u(t)$
based on the error $e(t) = r(t) - y(t)$, where $r(t)$ is the desired setpoint
and $y(t)$ is the measured state:
$$
u(t) = K_p e(t) + K_i \int_0^t e(\tau) d\tau + K_d \frac{de(t)}{dt}
$$
The Proportional ($K_p$) term responds to current error, the Integral ($K_i$)
term eliminates steady-state drift, and the Derivative ($K_d$) term
anticipates future error to dampen oscillations§. We designed
one controller for altitude ($z$) and three for attitude ($\phi, \theta, \psi$).

\subsection{Control Allocation}
The four PID outputs, known as virtual controls, $\mathbf{u}_v = [T, \tau_\phi, \tau_\theta, \tau_\psi]^T$,
must be mapped to the four individual motor thrusts $\mathbf{T}_m = [T_1, T_2, T_3, T_4]^T$.
Based on the quadcopter configuration in our proposal (a '+' configuration) and
assuming motor 1 is rear, 3 is front, 2 is right, and 4 is left:
\begin{align*}
T &= T_1 + T_2 + T_3 + T_4 \\
\tau_\phi &= L(T_2 - T_3) \quad \text{(Roll torque)} \\
\tau_\theta &= L(T_1 - T_4) \quad \text{(Pitch torque)} \\
\tau_\psi &= k_m(T_1 - T_2 + T_3 - T_4) \quad \text{(Yaw torque)}
\end{align*}
where $L$ is the arm length and $k_m$ is the thrust-to-torque coefficient.
This can be written as $\mathbf{u}_v = M \mathbf{T}_m$. For the controller,
we need the inverse mapping $\mathbf{T}_m = M^{-1} \mathbf{u}_v$:
$$
\begin{bmatrix} T_1 \\ T_2 \\ T_3 \\ T_4 \end{bmatrix}
=
\begin{bmatrix}
1/(4) & 0 & 1/(2L) & 1/(4k_m) \\
1/(4) & 1/(2L) & 0 & -1/(4k_m) \\
1/(4) & -1/(2L) & 0 & 1/(4k_m) \\
1/(4) & 0 & -1/(2L) & -1/(4k_m)
\end{bmatrix}
\begin{bmatrix} T \\ \tau_\phi \\ \tau_\theta \\ \tau_\psi \end{bmatrix}
$$
This $M^{-1}$ matrix is the control allocation matrix.

\subsection{Numerical Simulation}
The simulation was implemented in Python. We used a main loop that calls
\texttt{scipy.integrate.solve\_ivp} at each discrete control step ($dt=0.01s$).
At the start of each step, the PID controllers calculate the virtual controls $\mathbf{u}_v$
based on the current state $\mathbf{x}(t_i)$. These are mapped to motor thrusts
$\mathbf{T}_m$ which are held constant for the duration of the step.
\texttt{solve\_ivp} then numerically integrates the ODE system
$\dot{\mathbf{x}} = f(t, \mathbf{x}, \mathbf{T}_m)$ from $t_i$ to $t_{i+1}$.
This approach models a digital controller with a zero-order hold,
sampling the continuous-time dynamics of the quadcopter.


% --- 5. RESULTS ---
\section{Results}

\subsection{PID Tuning}
The PID gains $(K_p, K_i, K_d)$ for each of the four controllers were tuned
manually using a method based on Ziegler-Nichols. First, the $K_i$ and $K_d$
gains were set to zero. The proportional gain $K_p$ was then increased
until the system (e.g., the roll angle) exhibited stable, sustained
oscillations, noting this ultimate gain $K_u$ and oscillation period $T_u$.
The gains were then set using "classic PID" ratios: $K_p = 0.6 K_u$,
$K_i = 2 K_p / T_u$, and $K_d = K_p T_u / 8$. This process was repeated for
the roll and pitch controllers (which share gains due to symmetry), followed
by the yaw controller, and finally the altitude controller. The gains were
then fine-tuned to achieve a fast, critically damped response with
minimal overshoot.

% TODO: State your final Kp, Ki, Kd values in a table.

\subsection{Hover Stabilization}
% TODO: Insert your altitude plot here.
\begin{figure}[h!]
\centering
% \includegraphics[width=0.8\textwidth]{quadcopter_position.png}
\caption{Altitude (z-axis) response to a 1.0m step command.
The controller stabilizes the quadcopter at the desired altitude.}
\label{fig:altitude}
\end{figure}

\subsection{Disturbance Rejection}
% TODO: Insert your angle plot here.
\begin{figure}[h!]
\centering
% \includegraphics[width=0.8\textwidth]{quadcopter_attitude.png}
\caption{Attitude (roll/pitch) response to a simulated disturbance
(e.g., a 10-degree roll command for 1 second). The plot shows
the controller's ability to reject the disturbance and return to a
stable hover.}
\label{fig:attitude}
\end{figure}


% --- 6. DISCUSSION ---
\section{Discussion}
% TODO: Write this section.
% Analyze your plots from Section 5.
% Did your results match your expectations?
% What do the plots show? (e.g., "Figure 1 shows the controller
% successfully brings the quadcopter to a 1.0m hover, but with
% a 15% overshoot and a settling time of 2.5 seconds...")
% What were the limitations? (e.g., "We assumed a diagonal inertia
% tensor and ignored aerodynamic drag on the body... We also did not
% model motor saturation or time delays...")
% What is future work? (e.g., "Implement trajectory tracking...")


% --- 7. CONCLUSION ---
\section{Conclusion}
% TODO: Write a brief summary.
% (e.g., "We successfully modeled the 12-DOF non-linear dynamics
% of a quadcopter. We designed and tuned a PID control system
% that stabilized the inherently unstable system. Our simulation
% results show the controller can achieve a stable hover and
% reject external disturbances, confirming the validity of our
% model and control approach.")


% --- 8. REFERENCES ---
\printbibliography


% --- 9. APPENDIX (Optional) ---
\appendix
\section{Simulation Code}
\begin{verbatim}
import numpy as np
from scipy.integrate import solve_ivp
import matplotlib.pyplot as plt

# ... paste your full, final Python script here ...
# (See quadcopter_simulation.py)

\end{G}
\end{verbatim}

\end{document}