%%%%%%%%%%%%%%%%%%%%%%%%%%%%%%%%%%%%%%%%%%%%%%%%%
% MAT292 Final Project Report Draft
% Final Version - Tuned Gains and Methodology Included
%%%%%%%%%%%%%%%%%%%%%%%%%%%%%%%%%%%%%%%%%%%%%%%%%
\documentclass[12pt, letterpaper]{article}

% --- PACKAGES ---
\usepackage[margin=1in]{geometry} % 1-inch margins
\usepackage{amsmath} % For math equations
\usepackage{amssymb} % For \mathbb symbols
\usepackage{graphicx} % For inserting figures (remove [draft] once plots are saved)
\usepackage{hyperref} % For clickable links
\usepackage{booktabs} % For nice tables (in results)
\usepackage{indentfirst} % indent paragraphs
\usepackage[
backend=biber,
style=ieee, % A common engineering style for citations
sorting=none
]{biblatex}
\addbibresource{references.bib} % Use a separate .bib file

% --- TITLE INFO ---
\title{Attitude Control for Quadcopters: A PD-Based Approach for Stabilization}
\author{Akash Ajin, Akhilesh Menon, Paul Choi, Shiv Kanade\\ MAT292: Ordinary Differential Equations}
\date{Fall 2025}

\begin{document}

\maketitle


% --- 1. ABSTRACT ---
\begin{abstract}
This project investigates the attitude dynamics of a quadcopter, an inherently unstable system
that requires continuous feedback control for stable flight. We derive a 12-state nonlinear
rigid-body model based on the Newton--Euler equations of motion and formulate it as a coupled
system of ordinary differential equations (ODEs). Using this model, we design a
Proportional--Derivative (PD) control architecture for attitude (roll, pitch, yaw) and altitude. Numerical simulations using a
Runge--Kutta ODE solver (\texttt{scipy.integrate.solve\_ivp}) show that the tuned controllers
stabilize the vehicle from rest, track step commands in roll and yaw with small overshoot
and short settling times, and maintain altitude with negligible steady-state error. These
results illustrate how ODE modeling and numerical integration directly enable the analysis and
design of feedback controllers for real engineering systems.
\end{abstract}

% --- 2. INTRODUCTION ---
\section{Introduction}

\subsection{Motivation}
A quadcopter \setlength{\parindent}{24pt} is an unmanned aerial vehicle (UAV) whose flight is controlled by
four motors. This UAV's navigational agility has made it popular
in fields from photography to logistics. However, quadcopters are
inherently unstable. Without a constant stream of adjustments
from a control system, minor disturbances would cause them to tumble.
The system's dynamics are also highly non-linear and coupled, making it a
very fitting problem for an ODE course.

\subsection{Project Goal}

The primary objective of this project was to model the unstable flight dynamics
of a quadcopter, of mass 500 grams and arm length 0.225 meters, as a system of coupled, non-linear ordinary differential equations,
and to use this model to design and tune a stabilizing feedback controller.
We implement a PD architecture for attitude and altitude control,
and evaluate their performance in numerical simulation. The project provides a
clear, practical demonstration of how the tools of an ODE course (modeling,
linear algebra, and numerical solvers) are used to solve a fundamental problem
in modern robotics. In this work we restrict attention to hover and small-angle
attitude maneuvers, which allows us to focus on deriving and simulating the
underlying ODEs without the added complexity of full 3D trajectory tracking. Hover 
is treated as an equilibrium point of the nonlinear system, and the goal of the 
feedback controller is to locally stabilize this equilibrium in the presence of 
small disturbances. 
The central takeaway of this project is that modeling the quadcopter using the  
nonlinear equations of motion highlights coupling effects and behaviors 
that are not captured by simplified or linearized models about hover. In 
particular, the nonlinear model provides clearer insight into how attitude and 
altitude dynamics interact during disturbance rejection and controller response.
Controller performance is evaluated quantitatively in simulation using standard 
time-domain metrics such as rise time, settling time, and overshoot.



% --- 3. MATHEMATICAL MODEL ---
\section{Mathematical Model and Theoretical Foundation}

The quadcopter's motion is modeled as a rigid body in 3D space, using a
fixed inertial frame (Earth, $E$) and a rotating body frame (Body, $B$) attached to the vehicle~\cite{mahony2012multirotor,hoffmann2007quads}.

\subsection{State Vector}

The system is described by a 12-element state vector $\mathbf{x} \in \mathbb{R}^{12}$:
$$
\mathbf{x} = [
\underbrace{x, y, z}_{\substack{\text{position} \\ \text{(Frame E)}}},
\underbrace{\phi, \theta, \psi}_{\substack{\text{attitude (Euler)} \\ \text{(Frame E)}}},
\underbrace{\dot{x}, \dot{y}, \dot{z}}_{\substack{\text{lin. velocity} \\ \text{(Frame E)}}},
\underbrace{p, q, r}_{\substack{\text{ang. velocity} \\ \text{(Frame B)}}}
]^T
$$
This state is a hybrid, as linear motion is tracked in the inertial frame $E$
while rotational motion is tracked in the body frame $B$. This requires
transformation matrices to couple the dynamics.

\subsection{Translational Dynamics}

In the inertial frame, Newton's 2nd Law governs translational motion:
$$
\mathbf{F}_{\text{net}, E} = m \ddot{\mathbf{p}}_E = m \begin{bmatrix} \ddot{x} \\ \ddot{y} \\ \ddot{z} \end{bmatrix}
$$
The net force is the sum of gravity $\mathbf{F}_{grav, E} = [0, 0, -mg]^T$ and
the total thrust $\mathbf{F}_{thrust, B} = [0, 0, T]^T$. Thrust is generated
in the body frame (acting along the quadcopter's $z_B$-axis) and must be
rotated into the inertial frame:
$$
\begin{bmatrix} \ddot{x} \\ \ddot{y} \\ \ddot{z} \end{bmatrix} =
\frac{1}{m} \left(
\begin{bmatrix} 0 \\ 0 \\ -mg \end{bmatrix} +
R(\phi, \theta, \psi) \begin{bmatrix} 0 \\ 0 \\ T \end{bmatrix}
\right)
$$
where $T = T_1+T_2+T_3+T_4$, $m$ is the mass(0.5kg), and $R$ is the $Z-Y-X$ body-to-inertial rotation matrix.
$R$ is constructed as $R = R_z(\psi) R_y(\theta) R_x(\phi)$:
$$
R =
\begin{bmatrix}
c\psi c\theta & c\psi s\theta s\phi - s\psi c\phi & c\psi s\theta c\phi + s\psi s\phi \\
s\psi c\theta & s\psi s\theta s\phi + c\psi c\phi & s\psi s\theta c\phi - c\psi s\phi \\
-s\theta & c\theta s\phi & c\theta c\phi
\end{bmatrix}
$$
where $c\alpha = \cos(\alpha)$ and $s\alpha = \sin(\alpha)$.
The first six ODEs are thus: $\dot{x} = \dot{x}$, $\dot{y} = \dot{y}$, $\dot{z} = \dot{z}$,
and the three translational acceleration equations above ($\ddot{x}, \ddot{y}, \ddot{z}$).

\subsection{Rotational Dynamics}

Rotational motion is described by the Newton--Euler equations in the body frame~\cite{greenwood2003classical,mahony2012multirotor}:
$$
\mathbf{\tau}_{B} = I \dot{\mathbf{\omega}}_B + \mathbf{\omega}_B \times (I \mathbf{\omega}_B)
$$
where $\mathbf{\tau}_B = [\tau_\phi, \tau_\theta, \tau_\psi]^T$ are the net torques,
$I$ is the $3 \times 3$ inertia tensor, and $\mathbf{\omega}_B = [p, q, r]^T$.
Assuming a diagonal inertia tensor $I = \text{diag}(I_{xx}, I_{yy}, I_{zz})$,
the ODEs for the angular rates are:
\begin{align*}
\dot{p} &= (1/I_{xx}) \left( \tau_\phi - (I_{zz} - I_{yy})qr \right) \\
\dot{q} &= (1/I_{yy}) \left( \tau_\theta - (I_{xx} - I_{zz})pr \right) \\
\dot{r} &= (1/I_{zz}) \left( \tau_\psi - (I_{yy} - I_{xx})pq \right)
\end{align*}
The Euler angle derivatives (in frame E) are related to the body rates (in frame B)
by the transformation $W$:
$$
\begin{bmatrix} \dot{\phi} \\ \dot{\theta} \\ \dot{\psi} \end{bmatrix} =
W(\phi, \theta) \begin{bmatrix} p \\ q \\ r \end{bmatrix}
=
\begin{bmatrix}
1 & \sin\phi \tan\theta & \cos\phi \tan\theta \\
0 & \cos\phi & -\sin\phi \\
0 & \sin\phi / \cos\theta & \cos\phi / \cos\theta
\end{bmatrix}
\begin{bmatrix} p \\ q \\ r \end{bmatrix}
$$
These nine equations form the complete system of 12 first-order ODEs
($\dot{x}, \dot{y}, \dot{z}, \dot{\phi}, \dot{\theta}, \dot{\psi}, \ddot{x}, \ddot{y}, \ddot{z}, \dot{p}, \dot{q}, \dot{r}$),
which can be written in the form $\dot{\mathbf{x}} = f(t, \mathbf{x}, \mathbf{u})$.


% --- 4. METHODOLOGY ---
\section{Methodology}

\subsection{Overall Control Architecture}

The controller is organized in two layers. An \textbf{inner attitude loop} uses three
PD controllers to track desired Euler angles $(\phi_d, \theta_d, \psi_d)$ by
commanding body-frame torques $(\tau_\phi, \tau_\theta, \tau_\psi)$. An
\textbf{altitude loop} uses a PD controller to track the desired height $z_d$ by
commanding the total thrust $T$. Together, these four controllers produce the
virtual control vector
\[
    \mathbf{u}_v = [T, \tau_\phi, \tau_\theta, \tau_\psi]^T,
\]
which is then mapped to the individual motor thrusts via the inverse mixing
matrix $M^{-1}$ described below. In the simulations we chose simple step
commands (for example, $z_d = 1.0\,\text{m}$ and attitude steps
of $\phi_d = 10^\circ$, $\psi_d = 30^\circ$ over a finite time window), while
holding $\theta_d = 0^\circ$ to test cross-coupling.

\subsection{Control Strategy: PID Controller}

To stabilize the dynamics, we implemented four independent controllers
following the standard PID control law~\cite{astrom1995pid,ogata2010modern}, which computes a corrective action
\[
u(t) = K_p e(t) + K_i \int_0^t e(\tau)\,d\tau + K_d \frac{de(t)}{dt}.
\]
The proportional term $K_p e(t)$ responds to the current error, the integral
term accumulates past error to eliminate steady-state drift, and the derivative
term anticipates future error to damp oscillations. For all four axes
$(\phi, \theta, \psi, z)$ we ultimately set $K_i = 0$ and used a PD
structure, relying on gravity feed-forward compensation for the altitude
controller to eliminate steady-state error.

\subsection{Control Allocation}

The four PID outputs, known as virtual controls, $\mathbf{u}_v = [T, \tau_\phi, \tau_\theta, \tau_\psi]^T$,
must be mapped to the four individual motor thrusts $\mathbf{T}_m = [T_1, T_2, T_3, T_4]^T$.
Based on the quadcopter configuration and the final invertible mixing matrix used:
\begin{align*}
T &= T_1 + T_2 + T_3 + T_4 \\
\tau_\phi &= L(-T_1 + T_3) \quad \text{(Roll torque)} \\
\tau_\theta &= L(-T_2 + T_4) \quad \text{(Pitch torque)} \\
\tau_\psi &= k_m(-T_1 + T_2 - T_3 + T_4) \quad \text{(Yaw torque)}
\end{align*}
Where $k_m$ is the yaw torque coefficient (0.01 Nm/N), and $L$ is the arm length (0.225 m). This required the inverse mapping $\mathbf{T}_m = M^{-1} \mathbf{u}_v$, where $M^{-1}$ is:
$$
M^{-1} =
\begin{bmatrix}
1/4 & -1/(2L) & 0 & -1/(4k_m) \\
1/4 & 0 & -1/(2L) & 1/(4k_m) \\
1/4 & 1/(2L) & 0 & -1/(4k_m) \\
1/4 & 0 & 1/(2L) & 1/(4k_m)
\end{bmatrix}
$$

\subsection{Numerical Simulation}

The simulation was implemented in Python using the
    \texttt{scipy.integrate.solve\_ivp} function~\cite{virtanen2020scipy}, which employs a Runge--Kutta 45
method to solve the 12 ODEs~\cite{ascher1998computer}. We used a main loop that calls
    \texttt{solve\_ivp} at each discrete control step with sampling period
$\Delta t = 0.01\,\text{s}$. At each step $k$, the controller computes the
motor thrusts from the current state $\mathbf{x}(t_k)$, and
	\texttt{solve\_ivp} integrates the ODEs from $t_k$ to $t_{k+1} = t_k +
\Delta t$ using these thrusts as constant inputs. This mimics a digital flight
controller running at $100$\,Hz interacting with continuous-time vehicle
dynamics. The initial condition is a vehicle at rest near the origin, which
highlights the need for active stabilization.

% --- 5. RESULTS ---
\section{Results}

\subsection{PID Tuning}

The goal of tuning was to achieve a near \emph{critically damped} response for
all four axes: fast tracking with minimal overshoot or oscillation.

We initially attempted the Ziegler--Nichols Ultimate Cycle Method~\cite{ziegler1942optimum} to establish
a theoretical baseline. However, due to the highly non-linear nature and
cross-coupling present in the full ODE model, this approach yielded aggressive
gains that resulted in significant under-damped oscillations and, critically,
the integral term $K_i$ consistently triggered integral windup, leading to
instability in the attitude axes.

Therefore, the final stable design for the controllers uses a Proportional--Derivative (PD) structure
($K_i \approx 0$). The final tuned gains are
presented in Table~\ref{tab:gains}. For these gains, the altitude response
exhibits a rise time of roughly $t_r \approx 2.0\,\text{s}$ and settles
within a small band around the target by about $t_s \approx 3.0\,\text{s}$
with negligible overshoot. The roll and yaw responses have rise times on the
order of $1.0\,\text{s}$ and settle without sustained oscillations, which is
consistent with a nearly critically damped design.

\begin{table}[h!]
\centering
\caption{Final Controller Gains for Critically Damped Response}
\label{tab:gains}
\begin{tabular}{@{}lccc@{}}
\toprule
Controller & $K_p$ & $K_i$ & $K_d$ \\ \midrule
Altitude (z) & $3.00$ & $0.00$ & $2.280$ \\
Roll ($\phi$) & $0.08$ & $0.00$ & $0.043$ \\
Pitch ($\theta$) & $0.08$ & $0.00$ & $0.043$ \\
Yaw ($\psi$) & $1.00$ & $0.00$ & $0.057$ \\ \bottomrule
\end{tabular}
\end{table}

\subsection{Hover Stabilization}

The simulation commanded the quadcopter to maintain a hover at $z=1.0\text{m}$.
The Altitude controller performance is shown in Figure \ref{fig:altitude}.
\begin{figure}[h!]
\centering
\includegraphics[width=0.8\textwidth]{Position.png}
\caption{Altitude (z-axis) response to a 1.0m step command. The PD controller brings the quadcopter to the desired altitude in approximately 3 seconds with zero overshoot.}
\label{fig:altitude}
\end{figure}

\subsection{Attitude Disturbance Rejection}

The controller's ability to track commands and stabilize against disturbances was tested by commanding a 10-degree Roll and a 30-degree Yaw from $t=2\text{s}$ to $t=8\text{s}$. The attitude controller performance is shown in Figure \ref{fig:attitude}.
\begin{figure}[h!]
\centering
\includegraphics[width=0.8\textwidth]{Attitude.png}
\caption{Attitude (roll, pitch, yaw) response to simultaneous step commands (Roll=10$^{\circ}$, Yaw=30$^{\circ}$). The PD controller achieves fast rise time with minimal overshoot, and the Pitch axis remains stable at $0^{\circ}$, demonstrating excellent \textbf{cross-coupling rejection.}}
\label{fig:attitude}
\end{figure}


% --- 6. DISCUSSION ---
\section{Discussion}

The primary objective of stabilizing the quadcopter using an ODE-based model
and feedback control was successfully met.

\subsection{Interpretation of Results}

As shown in Figure~\ref{fig:altitude}, the altitude controller (PD) achieves
near-perfect tracking of the $z = 1.0\,\text{m}$ setpoint with no visible
overshoot and a short settling time. The gravity-compensation feed-forward
successfully eliminates the steady-state error that would otherwise be present due to
gravity and small modeling offsets.

Figure~\ref{fig:attitude} demonstrates the performance of the attitude
controllers (PD). The roll and yaw axes track their respective command steps
($10^{\circ}$ and $30^{\circ}$) with fast rise times and settle without the
destructive oscillations observed during the initial tuning phase. When
pitch is held at $0^{\circ}$, the response remains close to zero throughout
the maneuver, confirming the robustness of the mixing matrix $\mathbf{M}^{-1}$
and the effectiveness of the decoupled control design. Overall, the
time-domain behavior observed in simulation is consistent with the design
goal of a nearly critically damped response.

\subsection{Limitations and Future Work}

The model currently has several simplifying assumptions:
\begin{itemize}
    \item \textbf{No aerodynamic drag:} We assume drag on the body is
    negligible. This is accurate for low speeds but, at higher speeds,
    drag forces would need to be added to the translational ODEs in
    $\mathbf{F}_{\text{net},E}$, potentially changing the optimal gains.
    \item \textbf{Perfect control inputs:} We assume motors react instantly
    and linearly, ignoring actuator saturation and time delays. Including
    motor dynamics would require additional states and ODEs and would make the
    closed-loop system higher dimensional.
    \item \textbf{Attitude vs. position control:} The current controller
    regulates altitude and attitude but not horizontal position. A natural
    extension is a cascaded architecture where an outer-loop PID generates
    desired roll and pitch angles from $(x, y)$ position errors, enabling
    full 3D trajectory tracking.
\end{itemize}

These limitations suggest clear directions for future work: extend the ODE
model to include additional physics and actuators, and design corresponding
controllers that preserve stability and performance.

% --- 7. CONCLUSION ---
\section{Conclusion}

We derived a 12-DOF non-linear ODE system governing quadcopter dynamics and
implemented a stabilizing PD control architecture solved numerically with
Runge--Kutta methods (\texttt{scipy.integrate.solve\_ivp}). The simulations
show that the inherently unstable open-loop system can be stabilized, with
step responses that exhibit small overshoot and short settling times in both
altitude and attitude. This project demonstrates how the core ideas from an
ODE course, formulating physical laws as differential equations and solving
them numerically, translate directly into the analysis and design of
practical control systems~\cite{ogata2010modern,ascher1998computer}.


% --- 8. REFERENCES ---
\printbibliography


\end{document}